%% small-tree.tex
   %Copyright (C) 2019 N. Rochant
   
   % This program is free software: you can redistribute it and/or modify
	% it under the terms of the GNU General Public License as published by
    % the Free Software Foundation, either version 3 of the License, or
    % any later version.

    % This program is distributed in the hope that it will be useful,
    % but WITHOUT ANY WARRANTY; without even the implied warranty of
    % MERCHANTABILITY or FITNESS FOR A PARTICULAR PURPOSE.  See the
    % GNU General Public License for more details.

    % You should have received a copy of the GNU General Public License
    % along with this program.  If not, see <https://www.gnu.org/licenses/>.


\usepackage[edges]{forest}

\begin{figure}[H]
\centering
\caption{\small Andi within Nakh-Daghestanian}\label{andigenia}
\begin{forest}
for tree={
    edge path={
        \noexpand\path [thick, \forestoption{edge}] (!u.parent anchor) -- +(0,-7pt) -| (.child anchor)\forestoption{edge label};
    }, % this option together with setting the parent and child anchors to south and north, respectively, gives you the right-angle style for your branches
    parent anchor=south,
    child anchor=north,
    align=center, % allows you to put line breaks within nodes
}
[\textbf{Nakh-Daghestanian}, DarkBlue
	[Nakh]
    [\textbf{Daghestanian}, color=RoyalBlue4
		[Dargic]
		[Lezgic]
		[\textbf{Avar-Andic-Tsezic}, color=RoyalBlue4
			[Avar]
			[Tsezic]
			[\textbf{Andic}, color=RoyalBlue3
                [Akhvakh]
                [Bagwalal]
               	[Botlikh]                
               	[Godoberi]
               	[\textbf{Andi}, color=	RoyalBlue2
               	    [\textbf{Upper Andi},draw, \forestoption{edge}, color= OliveDrab3]
               	    [\textbf{Lower Andi},draw, \forestoption{edge}, color= MediumOrchid3]
               	]
                [Karata]           						[Chamalal]
				[Tindi]
        	]
		]
		[Lak]
        [Khinalug]
		]
]
\end{forest}
\end{figure}